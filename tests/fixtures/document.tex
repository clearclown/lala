\documentclass[12pt]{article}
\usepackage{amsmath}
\usepackage{amssymb}

\title{Sample LaTeX Document}
\author{Lala Editor Team}
\date{\today}

\begin{document}

\maketitle

\section{Introduction}

This is a sample LaTeX document for testing the \textbf{LaTeX preview} functionality in \emph{Lala Editor}.

LaTeX is widely used for typesetting mathematical formulas and scientific documents.

\section{Mathematical Expressions}

\subsection{Inline Math}

The famous equation $E = mc^2$ was discovered by Einstein.

The Pythagorean theorem states that $a^2 + b^2 = c^2$ for right triangles.

\subsection{Display Math}

The quadratic formula is:
$$
x = \frac{-b \pm \sqrt{b^2 - 4ac}}{2a}
$$

The Gaussian integral:
$$
\int_{-\infty}^{\infty} e^{-x^2} dx = \sqrt{\pi}
$$

\subsection{Equations}

\begin{equation}
\sum_{i=1}^{n} i = \frac{n(n+1)}{2}
\end{equation}

\begin{equation}
\nabla \cdot \mathbf{E} = \frac{\rho}{\epsilon_0}
\end{equation}

\section{Greek Letters and Symbols}

Common Greek letters: $\alpha, \beta, \gamma, \delta, \epsilon, \theta, \lambda, \mu, \pi, \sigma, \omega$

Uppercase: $\Gamma, \Delta, \Theta, \Lambda, \Pi, \Sigma, \Phi, \Psi, \Omega$

Mathematical operators: $\sum, \prod, \int, \partial, \nabla, \infty$

Relations: $\leq, \geq, \neq, \approx, \equiv, \propto$

Set theory: $\in, \notin, \subset, \supset, \cup, \cap, \emptyset$

Logic: $\forall, \exists, \neg, \wedge, \vee$

Arrows: $\rightarrow, \leftarrow, \Rightarrow, \Leftarrow, \Leftrightarrow$

\section{Lists}

\subsection{Itemize (Bullets)}

\begin{itemize}
\item First item with inline math: $x = 5$
\item Second item with Greek letter: $\alpha$
\item Third item with operator: $\sum_{i=1}^{n} x_i$
\end{itemize}

\subsection{Enumerate (Numbers)}

\begin{enumerate}
\item Define the problem
\item Formulate hypothesis
\item Conduct experiments
\item Analyze results
\item Draw conclusions
\end{enumerate}

\section{Conclusion}

This document demonstrates the LaTeX preview capabilities of Lala Editor.
The terminal renderer converts LaTeX commands to Unicode approximations.

For full compilation, use: \texttt{pdflatex document.tex}

\end{document}
